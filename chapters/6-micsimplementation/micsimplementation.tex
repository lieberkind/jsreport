\chapter{MiCS Implementation}
	In this chapter, the implementation of MiCS will be described. It will explain each of the five stages when generating JavaScript from C\# as explained in section \ref{sec:workflow_overview}. For clarity purposes, the third stage (Building + Mapping to Script\# AST) is divided into two different sections.


\section{Initializing MiCS} % (fold)
\label{sec:initializing_mics}





% section initializing_mics (end)

\section{Syntax Tree Validation} % (fold)
\label{sec:syntax_tree_validation}
	As we only support a fairly limited set of C\#’s built-in constructs and types, it is important to make sure that users only make use of those that we are able to map to Script\#. Should users utilise one of the constructs or types that we are not able to map, this should be pointed out with an understandable error message. It should not be left to the users to debug or understand a Roslyn or Script\# exception. Furthermore, it is important to make sure that users use their own code correctly. The remainder of this section will describe how a Validator class is used to achieve this.

	As mentioned above, it is necessary to make sure that users only utilise C\# constructs (method declarations, various statements and expressions, etc.) that we can correctly map to Script\#. At the moment, this is handled by the Builder classes, and will be explained in a later section. However, optimally this should be the responsibility of the Validator class. How this can be achieved is described in future work. It should also be confirmed that users utilise types and their members correctly. This is handled by the validator class. It should be noted that currently only methods are supported as members.

	To explain correct usage of types and members, it is beneficial to divide them into two categories; the ones that are built into the .NET platform and the ones users define themselves. Understanding “correct usage of .NET built-in types and members” is straightforward; it means that users are only allowed to use the types and members that can be mapped correctly to Script\#. How this is achieved is described later in this section. However, “correct usage of the types defined by the users themselves” requires some explanation. For this, the MixedSide Principle is introduced.

	\subsection{The MixedSide Principle} % (fold)
	\label{sub:the_mixedside_principle}
		The MixedSide Principle is a constraint that the user has to adhere to in order for MiCS to function correctly. As explained earlier the user can write server side code, which is only meant to be run only on the server, mixed side code, which is meant to be run both on server and client side and client side code, which is only meant to be run client side. The MixedSide Principle describes a simple ruleset for the interaction between serverside, mixedside and clientside code.

		\begin{figure}[H]
			\begin{center}
				\centerline{\includegraphics[width=12cm]{resources/images/MixedSidePrinciple.png}}
			\end{center}
			\caption{Visualization of The MixedSide Principle}
			\label{fig:MixedSidePrinciple}
		\end{figure}

		Code annotated with the ClientSide attribute is only meant to be run on the client in form of generated JavaScript. Therefore, it should not make instances of, or calls to methods on, objects that exist only on serverside, as no JavaScript will be generated from server side code. Consequently, if client side code interacts with server side code, it will ultimately result in an error when the JavaScript is generated as some methods and classes will not have been generated. However, JavaScript will be generated from code annotated with either the ClientSide attribute or the MixedSide attribute, so method calls to these are perfectly legal.

		Code that is not annotated with any attributes is regarded as server side code. No JavaScript will be generated from server side code. Server side code should only make calls to other server side code, or to mixed side code.

		As shown in Figure~\ref{fig:MixedSidePrinciple}, MixedSide code should be available both to client side code and server side code. As no communication should happen between client side and server side code, code annotated with the MixedSide attribute should only be able to interact with other code that has also been annotated with the MixedSide attribute. 

		If the MixedSide Principle is not violated and only built-in types and members that can be mapped are used, the users’ code is valid.

	% subsection the_mixedside_principle (end)

	\subsection{Validating} % (fold)
	\label{sub:validating}

		There are essentially three situations in which it is necessary to verify correct usage of types and members. The first is object creation. When an instance of a type is created, it is necessary to check the type in question can be mapped. The second situation is when members on type instances are accessed. It is then necessary to check first if the type can be mapped, then if the type has a member corresponding to the one being accessed. The third is invocation of methods. It is then necessary to check if the invocation is done correctly, using the correct arguments and return type.

		The Validator class extends Roslyns SyntaxWalker class and it is thus able to traverse syntax nodes. The Validator takes a CompilationUnit which holds the code to be validated, a string containing an attribute name that decides what methods to validate, and a structure of types and members that the validated methods are allowed to use. It works by looking for classes in the CompilationUnit that contains methods annotated with the given attribute name and validates the body of these methods against the provided structure of members.

		The nature of the Validator requires the Syntax Tree to be validated twice. Once validating all the MixedSide methods against a structure containing all MixedSide types and their members, and once validating all the ClientSide methods against a structure containing all ClientSide types and members, MixedSide types and members and ScriptSharp DOM types and members. This is easily done by creating two instances of the Validator and validating them both, as shown in figure X. In the example, the ScriptSharp DOM types are already contained in the clientSideMembers. 

		\begin{figure}[H]
			\begin{center}
				\centerline{\includegraphics[width=14cm]{resources/images/validatorInitiation.png}}
			\end{center}
			\caption{Initiating the Syntax Tree Validation process}
			\label{validatorInitiation}
		\end{figure}

		The Validation process is best explained by looking at an example. Consider the situation showed in Figure \ref{fig:mixedSideValidationExample}. 

		\begin{figure}[H]
			\begin{center}
				\centerline{\includegraphics[width=14cm]{resources/images/MixedSideValidationExample.png}}
			\end{center}
			\caption{Initiating the Syntax Tree Validation process}
			\label{fig:mixedSideValidationExample}
		\end{figure}		
		
		The Validator traverses the CompilationUnit depth-first and discovers the Validator class. It then finds all of the Validator class's methods and loops through these to see if they have the MixedSide attribute. When a method annotated with the MixedSide attribute is found, the validator visits it straight away, as shown in Figure \ref{fig:ValidatorVisitClassDeclaration}. 

		\begin{figure}[H]
			\begin{center}
				\centerline{\includegraphics[width=14cm]{resources/images/ValidatorVisitClassDeclaration.png}}
			\end{center}
			\caption{Visiting a ClassDeclaration and deciding whether its methods should be validated or not}
			\label{fig:ValidatorVisitClassDeclaration}
		\end{figure}

		The first method visited is the \texttt{IsStringValid()} method. The first statement of the method contains an object creation expression and the Validator now needs to check if the object creation is legal. It is legal either if the created object is a supported core type, or if the object is a user defined MixedSide type (residing within the members structure). As the type exists in the allowed members structure (shown if \ref{fig:mixedSideValidationExample}) the object creation is legal. If it had not existed in the member structure, and had not been a supported core type, the MixedSide Principle would have been violated, and an exception of type \texttt{MixedSidePrincipleViolatedException} had been thrown.








	
	% subsection validating (end)
% section syntax_tree_validation (end)

\section{Mapping to Script\# AST} % (fold)
\label{sec:mapping_to_scriptsharp_ast}
	When converting the validated Roslyn AST to its corresponding Script\# AST, a logical division of the process has been made. 

	\begin{itemize}
		\item Mapping a Roslyn AST node to its equivalent Script\# AST node
		\item Building a Script\# AST from all the mapped nodes
	\end{itemize}

	It is important to realize that the mapping is a sub process of the building process. The mapping is discussed in this section. 

	\subsection{Expressions, Statements and Symbols} % (fold)
	\label{sub:subsection_mapping_to_scriptsharp_expressions_statements_and_symbols}
		The mapping of Expressions, Statements and Symbols is implemented in three classes (ExpressionMapper.cs, StatementMapper.cs and SymbolMapper.cs). The three classes are logically divided (and named) after the kind of Script\# AST node they map to. 

		\begin{figure}[H]
			\begin{center}
				\centerline{\includegraphics[width=14cm]{resources/images/MapperClasses.png}}
			\end{center}
			\caption{Classes that define extension methods for mapping Roslyn AST nodes to Script\# AST nodes.}
			\label{mapperClasses}
		\end{figure}

		Mapping of AST nodes is somewhat straightforward as most Roslyn AST nodes have an equivalent Script\# AST node. E.g. a Roslyn return statement node maps to a Script\# return statement etc. The three mapper classes define \texttt{Map(...)} extension methods to the Roslyn node objects they map from (see example figures \ref{returnStatementMap} and \ref{conditionaleExpressionMap}).

		\begin{figure}[H]
			\begin{center}
				\centerline{\includegraphics[width=14cm]{resources/images/ReturnStatementMap.png}}
			\end{center}
			\caption{Extension method that maps Roslyn return statement to Script\# return statement. Note that \texttt{SS} is the Script\# namespace}
			\label{returnStatementMap}
		\end{figure}

		\begin{figure}[H]
			\begin{center}
				\centerline{\includegraphics[width=14cm]{resources/images/ConditionalExpressionMap.png}}
			\end{center}
			\caption{Extension method that maps Roslyn conditional expression to Script\# conditional expression. Note that \texttt{SS} is the Script\# namespace}
			\label{conditionaleExpressionMap}
		\end{figure}
	% subsection subsection_mapping_to_scriptsharp_expressions_statements_and_symbols (end)

	\subsection{Type Mapping} % (fold)
	\label{sub:type_mapping}
		Roslyn type symbols are mapped to Script\# type symbols using the \texttt{SymbolMapper}’s \texttt{Map(...)} extension method created for the Roslyn \texttt{TypeSymbol}. The mapping of type symbols from Roslyn to Script\# can be somewhat confusing since there are the different kinds of types (illustrated in figure \ref{typesOverview}) and there are some special cases (such as \texttt{null} which is mapped to the type \texttt{Object} in JavaScript). To illustrate the functionality of the \texttt{TypeSymbol} \texttt{Map(...)} method we have created the flowchart in figure \ref{typeMappingFlowchart}.

			\begin{figure}
			\begin{center}
					\includegraphics[width=14cm]{resources/images/TypeMappingFlowchart.png}
					\end{center}
				\caption{The type mapping process performed in \texttt{SymbolMapper.Map(this TypeSymbol typeSymbol)} extension method.}
				\label{typeMappingFlowchart}
			\end{figure}


		\subsubsection{Core Type Mapping} % (fold)
		\label{subsub:core:type_mapping}
			In contrast to using Script\# in the original manner (where the Script\# core types are used when writing code instead of the .NET core types) our project only uses the Script\# core types when mapping to the Script\# JavaScript AST. For this reason the Script\# core types are handled by its own type manager class (\texttt{ScriptSharpTypeManager.cs}). From the \texttt{ScriptSharpTypeManager} the Roslyn type symbols are retrieved before they are mapped to Script\# type symbols needed when building the Script\# AST.

			Since the .NET core types are used when writing code with MiCS and since these are mapped to the Script\# core types a mapping specification is needed. To facilitate this mapping we have created some simple classes to hold the specification (\texttt{MiCSCoreMapping.cs}, \texttt{MiCSCoreTypeMapping.cs} and \texttt{MiCSCoreMemberMapping.cs}) which can be queried using LINQ. This mapping specification contains information on the core types that we currently support. If a core type is not described in the specification then it is not supported. If the core type is found in the specification then the same pattern applies for its members. If a type member is not found then it is not supported. The mapping specification also holds information on a member’s return type, number of arguments and the arguments’ types.

			\begin{figure}[H]
					\includegraphics[width=8cm]{resources/images/InitiationOfTypeMapping.png}
				\caption{Instantiation example of a single core type (System.String) mapping specification.}
				\label{coreTypeMapping}
			\end{figure}

			An example of a core type mapping is the C\# System.String (see figure \ref{coreTypeMapping}) type which is mapped to the Script\# defined \texttt{System.String}. We are only mapping two of the String type’s members. The field \texttt{Length} which is mapped to the Script\# \texttt{String} type’s \texttt{Length} field (which is the equivalent of the JavaScript \texttt{String} object's property \texttt{length}). The second member we map is the \texttt{IndexOf(Char char)} method that returns an int. There are other \texttt{IndexOf} methods that take multiple arguments or a single argument of a different type (than \texttt{Char}) but these are not mapped in our mapping specification.

			Apart from the regular core types (i.e. the core types defined in the .NET \texttt{mscorlib.dll}) Script\# defines additional types in its \texttt{mscorlib.dll} file. One example is the regular expression class \texttt{Regex} which is of special interest as it is used in our case study. Usually the \texttt{Regex} class is placed in the \texttt{System.Text.RegularExpressions} namespace. In the Script\# \texttt{mscorlib.dll} however \texttt{Regex} is placed in the \texttt{System} namespace. The core type mapping should also express such differences in namespaces.
		% subsection subsection_name (end)
	% subsection type_mapping (end)
% section mapping_to_scriptsharp_ast (end)

\section{Building the Script\# AST} % (fold)
\label{sec:building_the_scriptsharp_ast}
	Building the Script\# AST consists of taking the mapped AST nodes and putting them together to form the Script\# AST. The builder classes uses the Roslyn infrastructure by extending the \texttt{SyntaxWalker} class which makes them capable of traversing the Roslyn AST.
	\begin{figure}[H]
		\begin{center}
			\centerline{\includegraphics[width=16cm]{resources/images/BuilderClasses.png}}
		\end{center}
		\caption{Classes that build the Script\# AST by traversing the Roslyn AST and using the \texttt{Map(...)} extension methods.}
		\label{builderClasses}
	\end{figure}

	The \texttt{NamespaceBuilder} class is responsible for building all of the types contained in a namespace. This is done by instantiating a \texttt{ClassBuilder} which in turn is responsible for building all of the member methods in a class (which is done using a \texttt{MethodBuilder}). This implies that the building of the Script\# AST is done in a depth first manner (the same way the Roslyn AST is traversed).  

	\begin{figure}[H]
		\begin{center}
			\centerline{\includegraphics[width=16cm]{resources/images/VisitClassDeclaration.png}}
		\end{center}
		\caption{Empty Script\# class is created and then built by using a \texttt{MethodBuilder} to retrieve all its member methods. The \texttt{ssClasses} property on the class builder holds all the classes that will be returned to the \texttt{NamespaceBuilder} who created it.}
		\label{visitClassDeclaration}
	\end{figure}

	The \texttt{NamespaceBuilder} and \texttt{ClassBuilder} classes are somewhat trivial. The \texttt{ClassBuilder} however ensures that DOM or core types are not be mapped to Script\#. The DOM and core types doesn't need to be generated as JavaScript types because they are built in types that already exists in JavaScript.

	The \texttt{MethodBuilder} class is a little more complex. It needs to build only the methods that are mixed side or client side methods. Furthermore it needs to handle a method’s return type, arguments and body statements. 

	Likewise the \texttt{StatementBuilder} and \texttt{ExpressionBuilder} are somewhat complex as building compound statements and expressions also bares some complexity. Before a compound statement or expression can be built all its child nodes and their associated types (if any) needs to be mapped. Figure \ref{visitIfStatement} shows an if-statement AST node is built.

	\begin{figure}[H]
		\begin{center}
			\centerline{\includegraphics[width=16cm]{resources/images/VisitIfStatement.png}}
		\end{center}
		\caption{To build an if-statement node its child nodes; condition, if-block and else-block (if any) has to be built first. Note that \texttt{SS} is the Script\# namespace}
		\label{visitIfStatement}
	\end{figure}

	When a builder class is done building the node(s) are returned to the parent builder class. Once all the namespaces has been built these constitute the Script\# AST which is then passed on in the overall workflow (to script generation).
% section building_the_scriptsharp_ast (end)

\section{Script Generation} % (fold)
\label{sec:script_generation}
	Script generation is done using the Script\# infrastructure only. Specifically the \texttt{TypeGenerator} class located in the \texttt{ScriptSharp.Generator} namespace is used for this. The \texttt{MiCSManager} is responsible for instantiating the \texttt{TypeGenerator} and providing it with types from the Script\# AST. This happens in the \texttt{MiCSManager.GenerateScriptText} method.
% section script_generation (end)

\section{Integration with Web Forms} % (fold)
\label{sec:integration_with_web_forms}
	Because of the time constraint on this project integration with Web Forms has only received a minimum amount of our time. We have ensured that we could implement our case study however there should be made improvements to the Web Forms integration code which we will discuss in section \ref{ssub:integration_with_web_forms}.

	However integration with Web Forms is currently done using the \texttt{MiCSPage} class. The \texttt{MiCSPage} class inherits from the \texttt{System.Web.UI.Page} class so that it’s possible to inherit from the \texttt{MiCSPage} class in one's Code Behind file.  The \texttt{MiCSPage} class searches the specific project's file structure for C\# files, reads their content and passes this content to the \texttt{MiCSManager} who then starts the entire MiCS workflow so that the client side script will be generated. The generated script is then registered with a \texttt{ScriptManager} which is how scripts are embedded into a page from Code Behind.

	\subsection{JavaScript-HTML Consistency} % (fold)
	\label{sub:javascript_html_consistency}
	
	% subsection javascript_html_consistency (end)
	Another important aspect of integration with Web Forms is how the generated client side scripts are initially called. Client side scripts are usually initiated through DOM events such as \texttt{onload} and \texttt{onclick} which are triggered when a web page is loaded and when a button is clicked respectively. MiCS currently only supports registering a client side method on a buttons click event using our \texttt{Button} extension method \texttt{OnClientClick(…)}. However this is how JavaScript-HTML consistency is guaranteed which is possible because the \texttt{Button} Web Control exists in the same C\# compile time validated environment (Code Behind) as the mixed side code and client side code does.

% section integration_with_web_forms (end)
