\chapter{MiCS Implementation}
	... indledning ...


\section{Types in MiCS} % (fold)
\label{sec:types_in_mics}
	To help understand the core type validation and MiCS type mapping in general its beneficial to realise the different kind of types that are utilized in MiCS.

	\begin{figure}[H]
		\begin{center}
			\centerline{\includegraphics[width=16cm]{resources/images/TypesOverview.png}}
		\end{center}
		\caption{Illustrates the different types used by MiCS.}
		\label{typesOverview}
	\end{figure}

	Since one of the goals of MiCS is to be able to execute the same code on both client and server side (server client portability) its required that the .NET core types are used when writing MiCS code. This is in contrast to how Script\# works in its original manner where the Script\# core types (that reflect the equivalent JavaScript types) are used. This has some benefits but is also an obstacle that prevents server client portability.

	\subsection{Core Types} % (fold)
	\label{sub:core_types}
		To build the ScriptSharp AST correctly the ScriptSharp core types are required to be associated to the AST nodes. One reason why the ScriptSharp core types are required is that they define their equivalent script name (in the class attributes) that is used by the ScriptSharp script generator. An example is the System.Char (see figure \ref{char}) type which is converted to the JavaScript String type as no JavaScript Char type exists.

	\begin{figure}[H]
			\includegraphics[width=13cm]{resources/images/Char.png}
		\caption{The core type System.Char defined in the Script\# mscorlib.dll.}
		\label{char}
	\end{figure}

		MiCS utilizes the regular .NET core types when a user is writing MiCS code but when generating the client side script the Script\# defined core types are used. This implies that some kind of mapping between the two kinds of core types are required. This mapping of core types is explained in section \ref{sub:type_mapping}.
	% subsection core_types (end)

	\subsection{User Types} % (fold)
	\label{sub:user_types}
		User types are the types that are defined by the MiCS end user. The user types considered here are either MixedSide types or ClientSide types (i.e. types that have method members that have the either the MixedSide attribute or the ClientSide on them). User type definitions is what the generated client side script eventually will consist of. 

		Pure server side types are obviously also user types but they are not that relevant in a MiCS context and therefore not discussed here.
	% subsection user_types (end)

	\subsection{DOM Types} % (fold)
	\label{sub:dom_types}
		Document Object Model (DOM) types are Script\# infrastructure defined in the System.Html namespace (Script.Web.dll). These classes that represent DOM objects from the browser. The purpose of these classes is only to represent the interface of the actual DOM types in the browser. This is also seen if one looks at the implementation of these types as all their methods and properties on these types return null or false. Like the Script\# core types, DOM types also has their script names in the attribute [ScriptName]. The DOM types are only meant for ClientSide code.

		\begin{figure}[H]
				\includegraphics[width=7cm]{resources/images/Document.png}
			\caption{Script\# definition of the DOM type Document.}
			\label{fig:document}
		\end{figure}
	% subsection dom_types (end)

% section types_in_mics (end)

\section{Initializing MiCS} % (fold)
\label{sec:initializing_mics}





% section initializing_mics (end)

\section{Syntax Tree Validation} % (fold)
\label{sec:syntax_tree_validation}
	As we only support a fairly limited set of C\#’s built-in constructs and types, it is important to make sure that users only make use of those that we are able to map to Script\#. Should users utilise one of the constructs or types that we are not able to map, this should be pointed out with an understandable error message. It should not be left to the users to debug or understand a Roslyn or Script\# exception. Furthermore, it is important to make sure that users use their own code correctly. The remainder of this section will describe how a Validator class is used to achieve this.

	As mentioned above, it is necessary to make sure that users only utilise C\# constructs (method declarations, various statements and expressions, etc.) that we can correctly map to Script\#. At the moment, this is handled by the Builder classes, and will be explained in a later section. However, optimally this should be the responsibility of the Validator class. How this can be achieved is described in future work. It should also be confirmed that users utilise types and their members correctly. This is handled by the validator class. It should be noted that currently only methods are supported as members.

	To explain correct usage of types and members, it is beneficial to divide them into two categories; the ones that are built into the .NET platform and the ones users define themselves. Understanding “correct usage of .NET built-in types and members” is straightforward; it means that users are only allowed to use the types and members that can be mapped correctly to Script\#. How this is achieved is described later in this section. However, “correct usage of the types defined by the users themselves” requires some explanation. For this, the MixedSide Principle is introduced.

	\subsection{The MixedSide Principle} % (fold)
	\label{sub:the_mixedside_principle}
		The MixedSide Principle is a constraint that the user has to adhere to in order for MiCS to function correctly. As explained earlier the user can write server side code, which is only meant to be run only on the server, mixed side code, which is meant to be run both on server and client side and client side code, which is only meant to be run client side. The MixedSide Principle describes a simple ruleset for the interaction between serverside, mixedside and clientside code.

		\begin{figure}[H]
			\begin{center}
				\centerline{\includegraphics[width=12cm]{resources/images/MixedSidePrinciple.png}}
			\end{center}
			\caption{Visualization of The MixedSide Principle}
			\label{fig:MixedSidePrinciple}
		\end{figure}

		Code annotated with the ClientSide attribute is only meant to be run on the client in form of generated JavaScript. Therefore, it should not make instances of, or calls to methods on, objects that exist only on serverside, as no JavaScript will be generated from server side code. Consequently, if client side code interacts with server side code, it will ultimately result in an error when the JavaScript is generated as some methods and classes will not have been generated. However, JavaScript will be generated from code annotated with either the ClientSide attribute or the MixedSide attribute, so method calls to these are perfectly legal.

		Code that is not annotated with any attributes is regarded as server side code. No JavaScript will be generated from server side code. Server side code should only make calls to other server side code, or to mixed side code.

		As shown in Figure~\ref{fig:MixedSidePrinciple}, MixedSide code should be available both to client side code and server side code. As no communication should happen between client side and server side code, code annotated with the MixedSide attribute should only be able to interact with other code that has also been annotated with the MixedSide attribute. 

		If the MixedSide Principle is not violated and only built-in types and members that can be mapped are used, the users’ code is valid.

	% subsection the_mixedside_principle (end)

	\subsection{Validating} % (fold)
	\label{sub:validating}

		There are essentially three situations in which it is necessary to verify correct usage of types and members. The first is object creation. When an instance of a type is created, it is necessary to check the type in question can be mapped. The second situation is when members on type instances are accessed. It is then necessary to check first if the type can be mapped, then if the type has a member corresponding to the one being accessed. The third is invocation of methods. It is then necessary to check if the invocation is done correctly, using the correct arguments and return type.

		The Validator class extends Roslyns SyntaxWalker class and it is thus able to traverse syntax nodes. The Validator takes a CompilationUnit which holds the code to be validated, a string containing an attribute name that decides what methods to validate, and a structure of types and members that the validated methods are allowed to use. It works by looking for classes in the CompilationUnit that contains methods annotated with the given attribute name and validates the body of these methods against the provided structure of members.

		The nature of the Validator requires the Syntax Tree to be validated twice. Once validating all the MixedSide methods against a structure containing all MixedSide types and their members, and once validating all the ClientSide methods against a structure containing all ClientSide types and members, MixedSide types and members and ScriptSharp DOM types and members. This is easily done by creating two instances of the Validator and validating them both, as shown in figure X. In the example, the ScriptSharp DOM types are already contained in the clientSideMembers. 

		\begin{figure}[H]
			\begin{center}
				\centerline{\includegraphics[width=14cm]{resources/images/validatorInitiation.png}}
			\end{center}
			\caption{Initiating the Syntax Tree Validation process}
			\label{validatorInitiation}
		\end{figure}

		The Validation process is best explained by looking at an example. Consider the situation showed in Figure \ref{fig:mixedSideValidationExample}. 

		\begin{figure}[H]
			\begin{center}
				\centerline{\includegraphics[width=14cm]{resources/images/MixedSideValidationExample.png}}
			\end{center}
			\caption{Initiating the Syntax Tree Validation process}
			\label{fig:mixedSideValidationExample}
		\end{figure}		
		
		The Validator traverses the CompilationUnit depth-first and discovers the Validator class. It then finds all of the Validator class's methods and loops through these to see if they have the MixedSide attribute. When a method annotated with the MixedSide attribute is found, the validator visits it straight away, as shown in Figure \ref{fig:ValidatorVisitClassDeclaration}. 

		\begin{figure}[H]
			\begin{center}
				\centerline{\includegraphics[width=14cm]{resources/images/ValidatorVisitClassDeclaration.png}}
			\end{center}
			\caption{Visiting a ClassDeclaration and deciding whether its methods should be validated or not}
			\label{fig:ValidatorVisitClassDeclaration}
		\end{figure}

		The first method visited is the \texttt{IsStringValid()} method. The first statement of the method contains an object creation expression and the Validator now needs to check if the object creation is legal. It is legal either if the created object is a supported core type, or if the object is a user defined MixedSide type (residing within the members structure). As the type exists in the allowed members structure (shown if \ref{fig:mixedSideValidationExample}) the object creation is legal. If it had not existed in the member structure, and had not been a supported core type, the MixedSide Principle would have been violated, and an exception of type \texttt{MixedSidePrincipleViolatedException} had been thrown.








	
	% subsection validating (end)
% section syntax_tree_validation (end)

\section{Mapping to ScriptSharp AST} % (fold)
\label{sec:mapping_to_scriptsharp_ast}
	When converting the validated Roslyn AST to the ScriptSharp AST we have made a logical division of the process. First the actual mapping from one Roslyn AST node to the equivalent ScriptSharp AST node. Secondly building the ScriptSharp AST from all the mapped nodes. The mapping is discussed in this section. 

	\subsection{Mapping to ScriptSharp Expressions, Statements and Symbols} % (fold)
	\label{sub:subsection_mapping_to_scriptsharp_expressions_statements_and_symbols}
		The mapping of Expressions, Statements and Symbols is implemented in three classes (ExpressionMapper.cs, StatementMapper.cs and SymbolMapper.cs). The three classes are logically divided (and named) after the type of ScriptSharp AST object they map to. 

		\begin{figure}[H]
			\begin{center}
				\centerline{\includegraphics[width=14cm]{resources/images/MapperClasses.png}}
			\end{center}
			\caption{Classes that define extension methods for mapping Roslyn AST nodes to Script\# AST nodes.}
			\label{mapperClasses}
		\end{figure}

		The mapping of the AST nodes is somewhat straightforward as most of the mappings we have done the Roslyn AST maps one to one with the ScriptSharp AST. So a Roslyn return statement node maps to a ScriptSharp return statement etc. The three classes define Map(...) extension methods to the Roslyn objects they map from (see example figures \ref{returnStatementMap} and \ref{conditionaleExpressionMap}).

		\begin{figure}[H]
			\begin{center}
				\centerline{\includegraphics[width=14cm]{resources/images/ReturnStatementMap.png}}
			\end{center}
			\caption{Extension method that maps Roslyn return statement to Script\# return statement.}
			\label{returnStatementMap}
		\end{figure}

		\begin{figure}[H]
			\begin{center}
				\centerline{\includegraphics[width=14cm]{resources/images/ConditionalExpressionMap.png}}
			\end{center}
			\caption{Extension method that maps Roslyn conditional expression to Script\# conditional expression.}
			\label{conditionaleExpressionMap}
		\end{figure}
	% subsection subsection_mapping_to_scriptsharp_expressions_statements_and_symbols (end)

	\subsection{Type Mapping} % (fold)
	\label{sub:type_mapping}
		Roslyn type symbols are mapped to Script\# type symbols using the SymbolMapper’s Map(...) extension method created for the Roslyn TypeSymbol. The mapping of TypeSymbols from Roslyn to Script\# can be somewhat confusing since there are the different kinds of types (illustrated in figure \ref{typesOverview}) and there are some special cases (such as null which is mapped to the type Object in JavaScript). To illustrate the functionality of the TypeSymbol Map(...) method we have created the flowchart in figure \ref{typeMappingFlowchart}.

			\begin{figure}
			\begin{center}
					\includegraphics[width=14cm]{resources/images/TypeMappingFlowchart.png}
					\end{center}
				\caption{Illustration of the type mapping process performed in SymbolMapper.Map(this TypeSymbol typeSymbol) extension method.}
				\label{typeMappingFlowchart}
			\end{figure}


		\subsubsection{Core Type Mapping} % (fold)
		\label{subsub:core:type_mapping}
			In contrast to using Script\# in the original manner (where the Script\# core types are used when writing code instead of the .NET core types) our project only uses the Script\# core types when mapping to the Script\# JavaScript AST. For this reason the Script\# core types are handled by its own type manager class (ScriptSharpTypeManager.cs). The Script\# core types’ source code are loaded into their own SemanticModel on the TypeManager class. From here the (Roslyn) types are retrieved before they are mapped to Script\# TypeSymbols needed when building the Script\# AST.

			Since the .NET core types are used when writing MiCS code and since these are mapped to the ScriptSharp core types a mapping specification is needed. To facilitate this mapping we have created some simple classes to hold the specification (MiCSCoreMapping.cs, MiCSCoreTypeMapping.cs and MiCSCoreMemberMapping.cs) which then can queried using LINQ. This mapping specification contains information on the core types that we currently support. So if a core type is not described in the specification then it is not supported. If the core type is found in the specification then the same pattern applies for its members. If a type member is not found then it is not supported. The mapping specification also holds information on a member’s return type, number of arguments and the arguments’ types.

			\begin{figure}[H]
					\includegraphics[width=8cm]{resources/images/InitiationOfTypeMapping.png}
				\caption{Instantiation example of a single core type (System.String) mapping specification.}
				\label{coreTypeMapping}
			\end{figure}

			An example of a core type mapping is the C\# System.String (see figure \ref{coreTypeMapping}) type which is mapped to the ScriptSharp defined System.String. We are only mapping two of the String type’s members. The field Length which is mapped to the ScriptSharp String type’s Length field (which is the equivalent of the JavaScript String object property length). The second member we map is the IndexOf(Char char) method that returns an int. There are other IndexOf methods that take multiple arguments or a single argument of a different type (than Char) but these are not mapped in our mapping specification.
		% subsection subsection_name (end)
	% subsection type_mapping (end)
% section mapping_to_scriptsharp_ast (end)

\section{Building the ScriptSharp AST} % (fold)
\label{sec:building_the_scriptsharp_ast}
	Building the Script\# AST consists of taking the mapped (Script\#) AST nodes and putting them together to form the Script\# AST. The builder classes utilize the Roslyn infrastructure by extending the SyntaxWalker class which makes them capable of traversing the Roslyn AST.
	\begin{figure}[H]
		\begin{center}
			\centerline{\includegraphics[width=16cm]{resources/images/BuilderClasses.png}}
		\end{center}
		\caption{Classes that built the Script\# AST by traversing the Roslyn AST and utilizing Map(...) extension methods.}
		\label{builderClasses}
	\end{figure}

	The NamespaceBuilder class is responsible for building all of its defined types. This is done by instantiating a ClassBuilder which in turn is responsible for building all of its member methods (which is done by instantiating a MethodBuilder). This implies that the building of the Script\# AST is done in a depth first manner. The NamespaceBuilder and ClassBuilder classes are somewhat trivial. 

	\begin{figure}[H]
		\begin{center}
			\centerline{\includegraphics[width=16cm]{resources/images/VisitClassDeclaration.png}}
		\end{center}
		\caption{Empty Script\# class is created and then built by using a MethodBuilder to retrieve all its member methods. The ssClasses property on the class builder holds all the classes that will be returned to the NamespaceBuilder who created it.}
		\label{visitClassDeclaration}
	\end{figure}

	The ClassBuilder has an important feature; it ensures that only user defined types will be mapped to the Script\# AST (and generated as script types). DOM (or core) types doesn’t need to be defined in Script\# as these obviously already exists in JavaScript (and thus not need to be generated again).

	The MethodBuilder class is a little more complex as it needs to build only the methods that are MixedSide or ClientSide methods. Furthermore it needs to handle a method’s return type, arguments and body statements. The StatementBuilder and ExpressionBuilder is however the most complex as building compound statements and expressions are more complicated. Before a compound statement or expression can be build all the child nodes and their associated types (if any) needs to be mapped.

	\begin{figure}[H]
		\begin{center}
			\centerline{\includegraphics[width=16cm]{resources/images/VisitIfStatement.png}}
		\end{center}
		\caption{To built an IfStatement node its child nodes; condition, if-block and else-block (if any) has to be built first.}
		\label{visitIfStatement}
	\end{figure}

	When a builder class is done building the node(s) are returned to the parent builder class. Once all the namespaces has been built these constitute the Script\# AST which is then passed on in the overall workflow (to script generation).
% section building_the_scriptsharp_ast (end)

\section{Script Generation} % (fold)
\label{sec:script_generation}
	Script generation is done using the ScriptSharp infrastructure only. Specifically the TypeGenerator class located in the ScriptSharp.Generator namespace is utilized for this. The MiCSManager is responsible for instantiating the TypeGenerator and providing it with the ScriptSharp JavaScript AST type nodes (this happens in the MiCSManager.GenerateScriptText method). The ScriptSharp JavaScript AST consists only of the user defined script types.
% section script_generation (end)

\section{Integration with Web Forms} % (fold)
\label{sec:integration_with_web_forms}
	sdsadq qdqw qwd qwd qw 
% section integration_with_web_forms (end)
