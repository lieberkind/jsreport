\section{Initializing MiCS} % (fold)
\label{sec:initializing_mics}

In chapter \ref{chap:mics_manual} it was shown how the end user only needs to make use of the \texttt{MiCSPage} class (instead of the \texttt{System.Web.UI.Page}) in order to utilize MiCS. This section describes what \texttt{MiCSPage} is doing behind the scenes in order to initialize MiCS.

The MiCSPage essentially has two jobs; gathering the developer's source code, and initiating the MiCSManager. It is then the MiCSManagers responsibility to build the Roslyn AST, initialize the necessary ressources required by MiCS and subsequently start the validation and building process. The ressources necessary for MiCS are essentially the type managers. As shown in figure \ref{fig:dependencygraph}, the type managers are used by almost all parts of MiCS.

The \texttt{MiCSPage} class inherits from the \texttt{System.Web.UI.Page} class. When the \texttt{OnPreRenderComplete} event is raised, \texttt{MiCSPage} uses the \texttt{System.IO} API to find all the C\# files in the current project and aggregates the files into a single source string. This source string is then handed to the \texttt{Initiate} method on the MiCSManager as shown in figure \ref{fig:init_mics_gather_source_code}.

\begin{figure}[H]
\begin{lstlisting}[language=CSharp,classoffset=1,morekeywords={Directory,SearcOption,File,MiCSManager,ScriptManager}]
foreach (string file in Directory.EnumerateFiles(rootPath, "*.cs", SearchOption.AllDirectories))
{
  source += File.ReadAllText(file);
}

MiCSManager.Initiate(source);
\end{lstlisting}
\caption{Gathering all C\# source code from the user's project}
\label{fig:init_mics_gather_source_code}
\end{figure}

The MiCSManager now first builds the Roslyn AST by using the \texttt{SyntaxTree.ParseText} method. When this is done, the MiCSManager initiates the TypeManager as shown in figure \ref{fig:init_mics_init_typemanager}

\begin{figure}[H]
\begin{lstlisting}[language=CSharp,classoffset=1,morekeywords={TypeManager,SyntaxTree}]
var userTree = SyntaxTree.ParseText(source);
TypeManager.Initate(userTree);
\end{lstlisting}
\caption{Initiaing the TypeManager.}
\label{fig:init_mics_init_typemanager}
\end{figure}

% The CSharpTypeManager holds information about all C\# types that can be used by MiCS, including user types (ClientSide \& MixedSide), DOM Types and supported .NET types.

The \texttt{TypeManager} now initiates the \texttt{CSharpTypeManager} and the \texttt{ScriptSharpTypeManager}. As explained in section \ref{sec:architecture}, the \texttt{CSharpTypeManager} holds type information on the developer defined types, the .NET core types, and the DOM types.
 When creating the \texttt{CSharpTypeManager}'s semantic model it is therefore necessary to make sure that all of these types can be recognized when they are looked up. The developer defined types are already contained in the Roslyn AST, but the other types need to be added. The DOM types are added simply by creating a new Roslyn AST consisting of the source of the developer syntax tree combined with the source of the DOM types from Script\#, as shown in figure \ref{fig:init_mics_add_dom_types}

\begin{figure}[H]
\begin{lstlisting}[language=CSharp,classoffset=1,morekeywords={SyntaxTree}]
var tree = SyntaxTree.ParseText(userTree.GetText() + ScriptSharp.TextSources.Web.Text);
\end{lstlisting}
\caption{Creating a SyntaxTree containing both the user types and the DOM types}
\label{fig:init_mics_add_dom_types}
\end{figure}

The supported .NET types are added as a reference to the semantic model when it is created.


% When creating the C\# Semantic Model it is therefore necessary to make sure that these supported .NET types can be recognized when they are looked up. This is done by handing the correct assemblies to the semantic model upon creation, as shown in figure \ref{fig:init_mics_create_semantic_model}.

% \begin{figure}[H]
% \begin{lstlisting}[language=CSharp,classoffset=1,morekeywords={MetadataFileReference,Regex,String,Compilation}]
% var mscorlib = new MetadataFileReference(typeof(String).Assembly.Location);
% var systemTextRegularExpression = new MetadataFileReference(typeof(System.Text.RegularExpressions.Regex).Assembly.Location);

% var compilation = Compilation.Create("Compilation", syntaxTrees: new[] { tree }, references: new[] { mscorlib, SystemTextRegularExpression });
% SemanticModel = compilation.GetSemanticModel(tree);
% \end{lstlisting}
% \caption{Creating the C\# Semantic Model}
% \label{fig:init_mics_create_semantic_model}
% \end{figure}

The \texttt{ScriptSharpTypeManager} only holds the types defined in the Script\# mscorlib.dll and therefore when creating the Script\# Semantic Model no other assemblies are needed.

Now that the TypeManagers are initialized, MiCS is ready for use.













% As mentioned in section X, the \texttt{MiCSManager} responsibility is to initialize all the ressources needed by MiCS and subs

% Before the conversion from C\# to JavaScript can begin, the ressources required by MiCS, namely the TypeManager classes, must be initialized. 


% In chapter \ref{chap:mics_manual} it was shown how the end user only needs to make use of the \texttt{MiCSPage} class (instead of the \texttt{System.Web.UI.Page}) in order to utilize MiCS. This section describes what \texttt{MiCSPage} is doing behind the scenes in order to initialize MiCS.

% The MiCSPage hooks up to an OnPreRenderComplete





% Before the conversion from C# to JavaScript can begin, the TypeManager classes must be initialized, as they are used by essentially all the other classes in MiCS. 

% The CompilationUnit, which is the root of the Roslyn AST and represents the user's C# code. This is the code that is validated by the Validator and the code from which JavaScript is eventually generated.




% The initiation of the TypeManagers is started from the MiCSManager. 







% Før MiCS kan begynde at konvertere C# til JavaScript, skal de nødvendige ressourcer konstrueres. 
% Det er det første der gøres, inden validering, konvertering og injektion finder sted.
% De ressourcer som skal initialiserers, er TypeManager'ne.
% 	Indeholder CompilationUnit, Semantiske Modeller, Core Mappings, MixedSide members (Alt med MixedSide) og ClientSide members (Alt med ClientSide / DOM Types) 


% * How does the MiCS project know what code to generate JavaScript from? // Forklares i afsnit om Roslyn
% * Gathering server side source code (currently from filepaths). // Forklares i WorkFlow afsnit?
% * Adding ScriptSharp DOM classes to semantic model.
% * Adding C# msCoreLib (System) and System.Text.RegularExpressions namespace to C# semantic model.



% section initializing_mics (end)