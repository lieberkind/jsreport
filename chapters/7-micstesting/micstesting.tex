\chapter{MiCS Testing}
	The testing done in this project can be divided into two categories; MiCS itself and the case study code. Most of our testing is done on MiCS itself. We have used the Visual Studio built-in unit testing framework for all of our tests.
\section{MiCS} % (fold)
\label{sec:mics}
	While implementing MiCS we have also created multiple unit tests located in the MiCSTests project. These tests primarily cover the features that are required by our case study. Because of the time constraints on this project, the quality of the tests are somewhat varying and the real code coverage is not 100\%. However we have still made some tests on core types, DOM types, statements, expressions, symbols and syntax tree validation.

	We have not performed any formal tests on the validity and quality of the Script\# generated scripts. Here we have instead emphasized building a valid Script\# AST and then relied on the existing Script\# infrastructure to work correctly. The reason for this is again the time constraints and the scope of the project.
% section mics (end)
\section{User Code Testing} % (fold)
\label{sec:user_code_testing}
	Another benefit of being able to reuse server side code on client side is that you also have to write less tests. All the MixedSide code can be tested with the same unit testing framework (i.e. the framework used on server side). Apart from making less tests this might also in some situations make it possible to stay with a single unit testing framework instead of one for server side and another one for client side. Another benefit might be that the quality of the server side unit testing frameworks and tools are better than those made for the client side. Simply because they have been used and developed more than their JavaScript equivalents (todo: FIND BACKING FOR THIS COMMENT!).

	To illustrate this possibility we have also created a small unit testing project that tests the validity of our MixedSide case study code (specifically the Validator class). (todo: MAKE THIS TEST PROJECT)

% section user_code_testing (end)