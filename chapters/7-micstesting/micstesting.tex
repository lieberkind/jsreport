\chapter{MiCS Testing}
	The testing done in this project can be divided into two categories; testing MiCS itself and testing the case study code. Most of our testing is done on MiCS itself. We have used the Visual Studio built-in unit testing framework for all of our tests.
\section{MiCS} % (fold)
\label{sec:mics}
	While implementing MiCS we have also created multiple unit tests located in the \texttt{MiCSTests} project. These tests primarily cover the features that are required by our case study. Because of the time constraints on this project, the quality of the tests are somewhat varying and the real code coverage is not 100\%. However we have still managed to write some tests for core types, DOM types, statements, expressions, symbols and syntax tree validation.

	We have not performed any formal tests on the validity and quality of the Script\# generated JavaScript. Here we have instead emphasized building a valid Script\# AST and then relied on the existing Script\# infrastructure to work correctly. The reason for this is again the time constraints and the scope of the project.
% section mics (end)
\section{Case Study Testing} % (fold)
\label{sec:user_code_testing}
	A benefit of using mixed side code is that you also have to write less tests. All mixed side code can be tested with the unit testing framework originally used only for server side. Therefore, relying only on a single unit testing framework (instead of one for server side and another for client side) is possible in some situations.

	To illustrate this possibility we have also created a small unit testing project \texttt{MiCSCaseStudyTests} that tests the validity of our mixed side case study code (specifically the \texttt{Validator} class).

% section user_code_testing (end)