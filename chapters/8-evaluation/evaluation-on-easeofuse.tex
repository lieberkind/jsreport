\section{Ease of Use} % (fold)
\label{sec:ease_of_use}
	\subsection{Error messages} % (fold)
	\label{sub:evaluation_of_error_messages}
		Many things can go wrong when using Roslyn. We have been forced to debug a lot of exceptions thrown by Roslyn and as described in section \ref{sub:design_error_messages}, it should not be left to the end user to debug exceptions thrown by Roslyn. Therefore, we have sought to find the source of these errors and throw exceptions with error custom messages that hints to the user what might have gone wrong and what he can do to fix it. However, as we do not have control over what code the user might write, we cannot guarantee that all possible problems Roslyn might run into when validating the user's code or converting it to JavaScript has been covered. The errors we have covered are, not surprisingly, the ones that we have had to debug when implementing our case study. This includes errors that are thrown when users try to use expressions, statements or .NET core types that MiCS cannot map or when they violate the Mixed Side Principle.
	% subsection evaluation_of_error_messages (end)

	\subsection{Client Server confusion} % (fold)
	\label{sub:client_server_confusion}
	
	% subsection client_server_confusion (end)








	% \begin{itemize}
	% 	\item Elaborate on the user friendliness of error messages, why they are important
	% 	and to which degree we have implemented them.
	% 	\item Indlæringskurve. Tungen lige i munden. Client-Server confusion
	% \end{itemize}
% section ease_of_use (end)