\section{Reflection on Implementation} % (fold)
\label{sec:reflection_on_implementation}
Todo: Skriv noget om at dette afsnit bl.a.\ omhandler forbedringer af den nuværende løsning.

\subsection{Validation} % (fold)
\label{ssub:validation}
% TODO: Should this be mentioned? It is necessary to make sure that developer only uses C\# constructs (method declarations, various statements and expressions, etc.) that we can correctly map to Script\#. At the moment, this is handled by the Builder classes, and will be explained in a later section. However, optimally this should be the responsibility of the Validator class. How this can be achieved is described in future work.

As explained in section \ref{sec:syntax_tree_validation}, when validating the Roslyn AST a structure of members is used to determine whether calls to types or methods defined by the developer are valid.\ This also applies to calls on DOM types and methods.\ While this solution works for our case study, it might not be appropriate as MiCS grows.\ Instead, the validator should make use of the Semantic Model on the CSharpTypeManager to look up types and methods. All the information that MiCS needs to decide whether the call is legal or not can be obtained from the semantic model. Furthermore, as the semantic model is automatically generated by Roslyn, it doesn't need to be maintained as MiCS grows which is not the case with the Collector class, that currently holds the responsibility for generating the members structure.
% subsection validation (end)

\subsection{Integration with Web Forms} % (fold)
\label{ssub:integration_with_web_forms}
Todo
% subsection integration_with_web_forms (end)

\subsection{Initializing MiCS} % (fold)
\label{ssub:collecting_source_code}
Todo
% subsection collecting_source_code (end)

\subsection{Extendability} % (fold)
\label{sub:extendability}

% subsection extendability (end)

% section reflection_on_implementation (end)