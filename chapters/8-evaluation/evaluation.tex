\chapter{Evaluation}

\section{Reflection on Case Study} % (fold)
\label{sec:reflection_on_case_study}
\begin{itemize}
	\item Comparison between MiCS and other form validation frameworks
\end{itemize}
% section reflection_on_case_study (end)

\section{Ease of Use} % (fold)
\label{sec:ease_of_use}

	\subsection{Error Messages} % (fold)
	\label{sub:error_messages}
		Elaborate on the user friendliness of error messages, why they are important
		and to which degree we have implemented them.	
	% subsection error_messages (end)

% section ease_of_use (end)

\section{Reflection on Implementation} % (fold)
\label{sec:reflection_on_implementation}
\subsection{Reflection on Validation} % (fold)
\label{sub:reflection_on_validation}
It is necessary to make sure that developer only uses C\# constructs (method declarations, various statements and expressions, etc.) that we can correctly map to Script\#. At the moment, this is handled by the Builder classes, and will be explained in a later section. However, optimally this should be the responsibility of the Validator class. How this can be achieved is described in future work.
% subsection reflection_on_validation (end)
% section reflection_on_implementation (end)

\section{Reflection on Design Goals} % (fold)
\label{sec:reflection_on_design_goals}

% section reflection_on_design_goals (end)

\section{Safety Benefits vs. Conveniente} % (fold)
\label{sec:safety_benefits_vs_conveniente}

\subsection{Pure ScriptSharp vs. MiCS} % (fold)
\label{sub:pure_scriptsharp_vs_mics}
More runtime errors with MiCS

% subsection pure_scriptsharp_vs_mics (end)

% section safety_benefits_vs_conveniente (end)
\section{Future Work}