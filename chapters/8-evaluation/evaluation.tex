\chapter{Evaluation}
	In this chapter we will evaluate on the different aspects of MiCS that we have described in the earlier chapters.

% Case study evaluation
\section{Case Study} % (fold)
\label{sec:reflection_on_case_study}

	We have had success in meeting the requirements set by our case study. All the necessary C\# constructs and types have been implemented and mapped to Script\# correctly, and the case study runs as expected; if the form does not validate on client side, an error message is displayed to the end user. If the validation passes on client side, the form is validated on server side using the same code as the JavaScript was generated from. If the client side validation is bypassed, e.g. by disabling JavaScript in the browser, the form is validated on server side. The full code needed to implement our case study is enclosed in appendix X.







	% \begin{itemize}
	% 	\item It's doable. Horray!
	% 	\item Comparison between MiCS and other form validation frameworks (Hvis der er tid)
	% \end{itemize}
% section reflection_on_case_study (end)

% Ease of use evaluation
\section{Ease of Use} % (fold)
\label{sec:ease_of_use}
	Activation of MiCS happens by inheriting from \texttt{MiCSPage} in one's Code Behind file (shown in figure \ref{fig:mics_enable_web_application}). This is a rather simple and easy approach in our opinion. When writing mixed side and client side code with MiCS the initial learning effort should be somewhat small. This is because the syntax written is that of C\# which is known to our target audience. Knowledge of regular JavaScript is however also required as objects representing DOM elements will also be used (specifically in client side only code). Since our target audience is used to developing web applications this should be a minor obstacle (if one at all) to overcome. Another aspect of ease of use is the fact that client side, mixed side and server side code can be written next to each other. This can be a bit confusing in some situation and might take some time to get used to. However it is still possible to have client side, mixed side and server side code separated in different files or classes which might help this issue. This in fact might be a good practice in any case. Overall we find the use of MiCS easy and well suited for our target audience.

	\subsection{Error messages} % (fold)
	\label{sub:evaluation_of_error_messages}
		We have been forced to debug a lot of exceptions thrown by Roslyn and as described in section \ref{sub:design_error_messages}, it should not be left to the end user to debug such exceptions. Therefore, we have sought to find the source of these errors and throw custom MiCS exceptions instead. These have been provided with custom error messages that hints to the developer what might have gone wrong and what he can do to fix the problem. However there might be Roslyn related problems that we have not considered, so the developer might still be exposed to Roslyn exceptions. This should obviously not happen as the developer will have no idea how the Roslyn exception is related to MiCS. This shortcoming of MiCS should preferably be fixed as it could make debugging for the developer very difficult. The errors we have covered are, not surprisingly, the ones that we have had to debug when implementing our case study. This includes errors that are thrown when a developer tries to use unsupported expressions, unsupported statements, C\# core types that MiCS cannot map or when the Mixed Side Principle is violated.
	% subsection evaluation_of_error_messages (end)

	% \label{sub:client_server_confusion}
	
	% subsection client_server_confusion (end)

% section ease_of_use (end)

% Implementation evaluation
\section{Reflection on Implementation} % (fold)
\label{sec:reflection_on_implementation}
Todo: Skriv noget om at dette afsnit bl.a.\ omhandler forbedringer af den nuværende løsning.

\subsection{Validation} % (fold)
\label{ssub:validation}
It is necessary to make sure that developer only uses C\# constructs (method declarations, various statements and expressions, etc.) that we can correctly map to Script\#. At the moment, this is handled by the Builder classes, and will be explained in a later section. However, optimally this should be the responsibility of the Validator class. How this can be achieved is described in future work.

As explained in section X, at the moment, when validating the Roslyn AST a members structure.
% subsection validation (end)

\subsection{Integration with Web Forms} % (fold)
\label{ssub:integration_with_web_forms}
Todo
% subsection integration_with_web_forms (end)

\subsection{Initializing MiCS} % (fold)
\label{ssub:collecting_source_code}
Todo
% subsection collecting_source_code (end)

\subsection{Extendability} % (fold)
\label{sub:extendability}

% subsection extendability (end)

% section reflection_on_implementation (end)

\section{Reflection on Server Client Consistency} % (fold)
\label{sec:reflection_on_server_client_consistency}
	Server client side consistency is especially relevant when embedding JavaScript from Code Behind (section \ref{sub:client_side_scripts_code_behind}) as this gives no guarantees if the function you are calling exists. When using a regular JavaScript editor such a problem will usually be highlighted for you however nothing prevents you from trying to run broken scripts. Running broken scripts by mistake is less likely with MiCS. If there is not consistency between server and client side an exception will be thrown. This is obviously a benefit if correctness of the web application is critical.
% section reflection_on_server_client_consistency (end)

\section{Runtime vs Compile Time Errors} % (fold)
\label{sec:runtime_vs_compile_time_errors}
	Compile time errors are usually preferred over runtime errors as they are easier to resolve and hence less time consuming. Furthermore runtime errors are potentially exposed to the end user.  When using Script\# as originally intended one gets the benefit of having your code compile time validated in the same manner as when using C\#. This is an important reason for choosing a Script\# only solution for some projects (see section \ref{sec:when_to_use_mics}).

	Code will also be compile time validated when using MiCS there are however some errors that can now occur as runtime errors on server side. This can e.g. happen when unsupported C\# constructs are used (TODO: REF) or when the Mixed Side Principle (section \ref{sub:the_mixedside_principle}) is violated. This is as disadvantage of using MiCS when compared to a Script\# only solution. However this disadvantage is not critical as it is isolated to server side which implies that the end user can be exempt from being exposed to these errors. Furthermore the server side runtime errors will still prevent a web application from running in a corrupt state (TODO: maybe REF). 

% section runtime_vs_compile_time_errors (end)

\section{Reflection on Mixed Side Code} % (fold)
\label{sec:reflection_on_mixed_side_code}
	Benefits to mixed side code are that existing server side code can easily be used for client side as well. So if you already have server side validation code it will work for client side as well which is obviously a great benefit as it can potentially safe you lots of time. Additionally when writing new code you will only have to write e.g. validation code once and therefore also only write tests once, which can obviously also safe you lots of time. This way Unit testing frameworks for server side will essentially be testing some of your client side code (i.e. mixed side code).

	One cost of mixed side code is that some compile time errors will become (server side) runtime errors as discussed in section \ref{sec:runtime_vs_compile_time_errors}. Another cost of mixed side code is that it can only contain C\# constructs that can be mapped to JavaScript. However this might not be that big a problem as most C\# constructs can be mapped to JavaScript. It might though in rare occasions limit the reuse of existing code or new mixed side code. However in most situations (see section \ref{sec:when_to_use_mics}) we find that the benefits outweighs the cost.

% section reflection_on_mixed_side_code (end)

\section{When to Use MiCS} % (fold)
\label{sec:when_to_use_mics}
	When choosing to use MiCS or a different safe JavaScript approach it’s beneficial to consider if one’s application is client side heavy, server side heavy or both. Using Script\# in the original manner would be a good approach to a client side heavy application as Script\# will reveal the most errors at compile time. 

	If you have an application that is both client side heavy and server side heavy using MiCS might be a good choice as it gives you the benefits of mixed side code and 'JavaScript HTML consistency' (section \ref{ssub:obtaining_javascript_html_consistency} TODO: UPDATE REF). The cost is that some of your compile time errors will become runtime errors (on server side) but this doesn’t necessary outweigh the other benefits. 

	If you have a web application that is server side heavy you might not use any framework for safe JavaScript development or you could use MiCS as the over head would be low.

	Another thing to consider when deciding to use MiCS is the likelihood of the server side implementation to change from ASP.NET Web Forms. When using MiCS the JavaScript is written on server side. If the server side implementation is to change, the JavaScript generated by MiCS has to be implemented ``manually'' (note that this also applies to the HTML pages generated from Code Behind). Using ASP.NET Web Form and MiCS creates a compile time validated web application, however, the coupling between the server and the client is very tight.

% section when_to_use_mics (end)


\section{Safety Benefits vs. Convenience} % (fold)
\label{sec:safety_benefits_vs_conveniente}
	Safety in development is the main target of this project however safety always comes at a cost. Unsafe JavaScript development can be done very fast and even though we think we have a solution to safe JavaScript development with very little over head unsafe JavaScript can be the right choice for some projects. However if a project reach a certain size or complexity (and if correctness matters) it will eventually be beneficial to use JavaScript in safe manner.


\section{Future Work}
\label{sec:futurework}
This section describes which direction the project should take from here; what needs improvements and which features should be implemented next.

\subsection{Testing} % (fold)
\label{sub:fw_testing}
	As stated in the introduction, this project has served as a proof of concept. Even though some testing has been done to secure the correctness of MiCS, it is not enough to guarantee that MiCS works correctly in all situations. Test coverage should be prioritized as MiCS ultimately should serve as a framework for other projects to build on.
% subsection fw_testing (end)

\subsection{Portability of Server Side Values} % (fold)
\label{sub:portability_of_server_side_values}
	Even though code server side code can be ported to JavaScript, in some situations the ability to port server side literals could be very handy. 
% subsection portability_of_server_side_values (end)

\subsection{Inheritance} % (fold)
\label{sub:fw_inheritance}
	Inheritance is a widely used feature of the C\# programming language and the .NET platform in general. Therefore, it should be prioritized to add support for the portability of inherited types.
% subsection fw_inheritance (end)

\subsection{Event Based Programming} % (fold)
\label{sub:fw_event_based_programming}
	A nice way to keep code decoupled is by using event based programming. Both C\# and JavaScript supports custom events, so support for this feature only seems natural.
% subsection fw_event_based_programming (end)

\subsection{Caching Generated Scripts} % (fold)
\label{sub:fw_script_caching}
	As of the moment, MiCS generates JavaScript on every page load. To optimize the loading time of every page request, it would be beneficial to investigate the opportunity of caching JavaScript source code that has already been generated.
% subsection fw_script_caching (end)

\begin{itemize}
	\item Todo: Modules for safe form registration
\end{itemize}
