
\section{Future Work}
\label{sec:futurework}
This section describes which direction the project should take from here; what needs improvements and which features should be implemented next.

\subsection{Testing} % (fold)
\label{sub:fw_testing}
	As stated in the introduction, this project has served as a proof of concept. Even though some testing has been done to secure the correctness of MiCS, it is not enough to guarantee that MiCS works correctly in all situations. MiCS ultimately should serve as a framework for other projects to build on thus test quality and coverage should be prioritized as correctness would be critical.
% subsection fw_testing (end)

\subsection{Completing JavaScript-HTML Consistency} % (fold)
\label{sub:completing_javascript_html_consistency}
	Since JavaScript-HTML consistency was not fully implemented in MiCS this would obviously be one of the first things to develop further. Currently only calls to the generated JavaScript from HTML elements (i.e. from the button’s client side click event) is compile time guaranteed. When JavaScript retrieves HTML elements by looking for an element with a specific ID there is no guarantee that such an HTML element exists. This guarantee constitutes the missing part of JavaScript-HTML consistency and is exemplified in figure \ref{fig:write_mics_code_clienside} where the HTML elements are retrieved using the hardcoded IDs (\texttt{EmailBox} and \texttt{CheckBox}). This is somewhat unsafe and a solution should therefore be implemented. This solution should built on the fact that a HTML representation exists in the same compile time validated environment (Code Behind) as the JavaScript representation (\texttt{ClientSide} code) does. This is also how the implemented part of JavaScript-HTML consistency is made possible.
% subsection completing_javascript_html_consistency (end)

\subsection{Inheritance} % (fold)
\label{sub:fw_inheritance}
	% Todo: Add to bug list
	Inheritance is a widely used feature of the C\# programming language and the .NET platform in general. Therefore, it should be prioritized to add support for the portability of inherited types. Currently inheritance is not handled at all in which is somewhat limiting. Apart from the developer not being able to create new types that use inheritance, inherited members on DOM types and core types cannot be used without first casting to the parent type. This should obviously be implemented at some point.
% subsection fw_inheritance (end)

\subsection{Event Based Programming} % (fold)
\label{sub:fw_event_based_programming}
	A nice way to keep code decoupled is by using event based programming. Both C\# and JavaScript supports custom events, so support for this feature only seems natural.
% subsection fw_event_based_programming (end)

\subsection{Caching Generated Scripts} % (fold)
\label{sub:fw_script_caching}
	As of the moment, MiCS generates JavaScript on every page load. To optimize the loading time of every page request, it would be beneficial to investigate the opportunity of caching JavaScript source code that has already been generated.
% subsection fw_script_caching (end)