\subsection{Microsoft Roslyn} % (fold)
\label{sub:microsoft_roslyn}
	Roslyn is a Microsoft project that exposes the C\# compiler as a service. There are three key features that makes Roslyn interesting.

  \subsubsection{Easy AST generation} % (fold)
  \label{ssub:easy_ast_generation}
  	First of all, with Roslyn it is easy to generate an AST (of type \texttt{SyntaxTree}) from C\# source code. This is done by using either the \texttt{ParseText} or the \texttt{ParseFile} method on Roslyn's \texttt{SyntaxTree} class, as shown in figure \ref{fig:roslyn_walker_example}. The generated AST consists of nodes of type \texttt{SyntaxNode}, which represent C\# constructs such as declarations, statements and expressions. E.g. an \texttt{if}-statement is represented by an \texttt{IfStatementSyntax} node and a binary expression, such as \texttt{1 + 2}, is represented by a \texttt{BinaryExpressionSyntax} node.
  % subsubsection easy_ast_generation (end)

  \subsubsection{Easy AST traversal} % (fold)
  \label{ssub:subsubsection_name}
  	Secondly, the generated AST can be traversed depth-first ealisy thanks to the fact that Roslyn's internal structure is based on the Visitor Pattern \cite{bib:visitorpattern}. By inheriting from Roslyn's \texttt{SyntaxWalker} class, all \texttt{SyntaxNode}s in a given \texttt{SyntaxTree} can be visited and processed as desired. This is done by overwriting the \texttt{SyntaxWalker}'s default \texttt{Visit}-methods. E.g. to visit an \texttt{IfStatementSyntax} node, the \texttt{SyntaxWalker}'s \texttt{VisitIfStatement}-method has to be overwritten. Figure \ref{fig:roslyn_walker_example} shows an example of a simlpe custom \texttt{SyntaxWalker} which prints the names of the namespaces and classes it visits.
  % subsubsection subsubsection_name (end)

	\begin{figure}
		\begin{center}
			
		\begin{lstlisting}[language=CSharp,classoffset=1,morekeywords={Program,SyntaxTree,TestWalker,Console,SyntaxWalker,NamespaceDeclarationSyntax,ClassDeclarationSyntax}]
namespace RoslynTest
{
  class Program
  {
    static void Main(string[] args)
    {
      var smallProgram = @"
        namespace TestNamespace 
        {
          class TestClass { }
        }
      ";

      SyntaxTree syntaxTree = SyntaxTree.ParseText(smallProgram);
      TestWalker testWalker = new TestWalker();

      testWalker.Visit(syntaxTree.GetRoot());
      Console.Read();
    }
  }

  class TestWalker : SyntaxWalker
  {
    public override void VisitNamespaceDeclaration(NamespaceDeclarationSyntax node)
    {
      string namespaceName = node.Name.ToString();
      Console.WriteLine("Hello, this is namespace {0}!", namespaceName);
      base.VisitNamespaceDeclaration(node);
    }

    public override void VisitClassDeclaration(ClassDeclarationSyntax node)
    {
      string className = node.Identifier.ValueText;
      Console.WriteLine("Hello, this is class {0}!", className);
      base.VisitClassDeclaration(node);
    }
  }
}



		\end{lstlisting}
		\end{center}
		\caption{Example of a custom SyntaxWalker}
		\label{fig:roslyn_walker_example}
	\end{figure}

  \subsubsection{Easy retrieval of type information} % (fold)
  \label{ssub:easy_retrieval_of_type_information}
  	Lastly, Roslyn can create a \emph{Semantic Model} for any generated AST. The Semantic Model functions as a refence table that contains information about the syntax nodes in the AST. For example, given an expression, the Semantic Model can determine its resultant type. Given an identifier, the Semantic Model can determine its type. This can be very useful, e.g. as many local variables across the source code can have the same name, but different types. So while the AST defines the program's syntactic structure, the Semantic Model helps identifying what is being referenced. The Semantic Model is not limited to looking up types defined within the AST. If external assemblies (such as mscorlib.dll) is handed to the Semantic Model upon creation, types within these assemblies can also be resolved.
  % subsubsection easy_retrieval_of_type_information (end)

\FloatBarrier
% subsection microsoft_roslyn (end)