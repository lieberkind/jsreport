\begin{abstract}
	This report investigates how to make the use of JavaScript safer when developing web applications with ASP.NET Web Forms. 

	Developing web applications instead of desktop applications is a strong tendency in software development today. JavaScript contributes to enriching the user experience and sometimes provide functionality which is indispensable for such web applications. However, JavaScript has language features that can be considered unsafe, such as no compile-time validation, a dynamic type system and implicit type conversions.

	This report investigates how to move JavaScript errors from client side to server side, how to generate JavaScript source code from code written in a server side language, and how to obtain a compile time validation of the consistency between the generated JavaScript and HTML.

	It is concluded that it is possible to generate JavaScript from C\# code by using a combination of Microsoft's Roslyn and Script\#, and that the consistency between the generated JavaScript and HTML can be compile time validated by extending the ASP.NET Web Forms framework.


\end{abstract}