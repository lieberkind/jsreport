\begin{abstract}
	Developing web applications instead of native applications is a strong tendency in software development today. JavaScript contributes to enriching the user experience and sometimes provides functionality which is indispensable for such web applications.

	JavaScript however, can in many ways be regarded as an unsafe programming language. Because of this many projects (e.g. Microsoft’s TypeScript or Google Web Toolkit) have built safe environments for writing client side scripts to help enforce correctness.

	To improve development of web applications with ASP.NET Web Forms from Code Behind this report investigates how JavaScript can be safely represented in order to acheive compile time validation. To improve safe development further the report investigates if consistency between client and server side can be guaranteed. Additionally the report investigates if making client and server side code portable between the two, as this would give additional safety and maintenance benefits.

	The report concludes...
\end{abstract}